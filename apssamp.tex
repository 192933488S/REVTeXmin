\part{title}%%%%%%%%%%%%%%%%%%%%%%%%%%%%%%%%%%%%%%%%%%%%%%%%%%%%%%%%%%%%%%%%%%%%%%%%%%%%%%%
%
%   This file is part of the APS files in the REVTeX 4.2 distribution.
%   Version 4.2a of REVTeX, December 2014
%
%   Copyright REVTeX (c) 2014 The American Physical Society.
%
% 	Build-Tool: TeXstudio 4.2.2
%   Configuration: Default; Try bibtex.exe ?a*.aux for Bibtex if needed
%
%%%%%%%%%%%%%%%%%%%%%%%%%%%%%%%%%%%%%%%%%%%%%%%%%%%%%%%%%%%%%%%%%%%%%%%%%%%%%%%
% 
% Usage
% 
% \pageref{}
% \begin{widetext}
% \end{widetext}
% \begin{quotation}\flushleft\leftskip1em
%	XXX\\
% \end{quotation}
% In Appendix:
%	\appendix
%	\section{}
% 	\begin{subequations}
%	\begin{eqnarray} % \nonumber for no number or \tag{} for other nums
%	\end{eqnarray}
%	\end{subequations}
% 
% % \begin{figure}
	% \includegraphics{}%
	% \caption{\label{}}
	% \end{figure}

% Surround figure environment with turnpage environment for landscape
% figure
% \begin{turnpage}
	% \begin{figure}
		% \includegraphics{}%
		% \caption{\label{}}
		% \end{figure}
	% \end{turnpage}
% 
% \begin{figure*}
% \includegraphics{fig_2}% Here is how to import EPS art
% \caption{\label{fig:wide}Use the figure* environment to get a wide
%	figure that spans the page in \texttt{twocolumn} formatting.}
% \end{figure*}
%
% Refer to auguide and summary for detailed author help and commands
%
%%%%%%%%%%%%%%%%%%%%%%%%%%%%%%%%%%%%%%%%%%%%%%%%%%%%%%%%%%%%%%%%%%%%%%%%%%%%%%%
\documentclass[%
preprint,
%reprint, %reprint for production %disable also mathlines package, see below; journal type should be selected
%superscriptaddress,
%groupedaddress,
%unsortedaddress,
%runinaddress,
%frontmatterverbose, 
showkeys,
%preprintnumbers,
nofootinbib,
%nobibnotes,
%bibnotes,
amsmath,
amssymb,
aps, %default
prl,
%pra,
%prb,
%rmp,
%prstab,
%prstper,
%floatfix,
]{revtex4-2}

\usepackage{graphicx}% Include figure files
\usepackage{dcolumn}% Align table columns on decimal point
\usepackage{bm}% bold math
\usepackage{hyperref}% add hypertext capabilities

% line look good in preprint, not reprint
\usepackage[mathlines]{lineno}% Enable numbering of text and display math
\linenumbers\relax % Commence numbering lines

%\usepackage[showframe,%Uncomment any one of the following lines to test 
%%scale=0.7, marginratio={1:1, 2:3}, ignoreall,% default settings
%%text={7in,10in},centering,
%%margin=1.5in,
%%total={6.5in,8.75in}, top=1.2in, left=0.9in, includefoot,
%%height=10in,a5paper,hmargin={3cm,0.8in},
%]{geometry}

% remove for APS submission (default req is US letter)
% check https://www.overleaf.com/learn/latex/Page_size_and_margins
\usepackage[a4paper, total={6in, 8in}]{geometry}


%%%%%%%%%%%%%%%%%%%%%%%%%%%%%%%%%%%%%%%%%%%%%%%%%%%%%%%%%%%%%%%%%%%%%%%%%%%%%%%
%
\usepackage[english, german]{babel}
% Use \selectlanguage{english} inside document environment
%\usepackage{lmodern} % Not needed, just install 'cm-super
% https://tex.stackexchange.com/questions/1291/why-are-bitmap-fonts-used-automatically/546882#546882
\usepackage[T1, T2A]{fontenc}
%
%%%%%%%%%%%%%%%%%%%%%%%%%%%%%%%%%%%%%%%%%%%%%%%%%%%%%%%%%%%%%%%%%%%%%%%%%%%%%%%

\begin{document}


\selectlanguage{english} 
\preprint{APS/123-QED}

\title{\textsc{APS} PRL apssamp.tex minimal}% Force line breaks with \\
\thanks{General introduction}%Only in reprint

\author{Your Name}
% \altaffiliation[Also at ]{Physics Department, XYZ University.}%Lines break automatically or can be forced with \\
%\author{Second Author}%
 \email{your.name@gmail.com}
\affiliation{%
 City,\\ Country
% This line break forced with \textbackslash\textbackslash
}%

\date{\today}% It is always \today, today,
             %  but any date may be explicitly specified

\begin{abstract}
An article usually includes an abstract, a concise summary of the work
covered at length in the main body of the article. % avoid citation in abstract
\begin{description}
\item[Usage]
Secondary publications and information retrieval purposes.
%\item[Structure]
%You may use the \texttt{description} environment to structure your abstract;
%use the optional argument of the \verb+\item+ command to give the category of each item. 
\end{description}
\end{abstract}

\keywords{Keyword 1, Keyword 2, Keyword 3}
\maketitle

%\tableofcontents

%%%%%%%%%%%%%%%%%%%%%%%%%%%%%%%%%%%%%%%%%%%%%%%%%%%%%%%%%%%%%%%%%%%%%%%%%%%%%%%

\section{\label{sec:level1}Section}

This is a very short sentence.\cite{zeilinger_experiment_1999}

\subsection{\label{sec:level2}Subsection}

This is a very short sentence.

\paragraph{Paragraph}
This is a very short sentence.


\subsubsection{Subsubsection}

\paragraph{Paragraph} % check how paragraph works and looks
AAA

BBB

CCC

\subsubsection{Subsubsection}

\paragraph{Paragraph}
AAA
BBB
CCC


%%%%%%%%%%%%%%%%%%%%%%%%%%%%%%%%%%%%%%%%%%%%%%%%%%%%%%%%%%%%%%%%%%%%%%%%%%%%%%%

\section{Maths}
%
% fleqn to flush equations left.
% $\bm{B}$
% $\mathfrak{B}$
% $\mathbb{B}$
% \[ XXX \] %unnumbered single-line eq

Below we have numbered single-line equations; this is the most common
type of equation in \textit{Physical Review}:
\begin{eqnarray}\label{eq1}
\bm{B} = 0\\
\mathfrak{B} = 0\\
\mathbb{B} = 0
\end{eqnarray}


\subsection{Subsection}

Multiline equations are obtained.


Enclosing display math within
\verb+\begin{subequations}+ and \verb+\end{subequations}+ will produce
a set of equations that are labeled with letters, as shown in
Eqs.~(\ref{subeq:1}) and (\ref{subeq:2}) below.
You may include any number of single-line and multiline equations,
although it is probably not a good idea to follow one display math
directly after another.
\begin{subequations}
\label{eq:whole}
\begin{eqnarray}
{\cal M}=&&ig_Z^2(4E_1E_2)^{1/2}(l_i^2)^{-1}
(g_{\sigma_2}^e)^2\chi_{-\sigma_2}(p_2)\nonumber\\
&&\times
[\epsilon_i]_{\sigma_1}\chi_{\sigma_1}(p_1).\label{subeq:2}
\end{eqnarray}
\begin{equation}
\left\{
 abc123456abcdef\alpha\beta\gamma\delta1234556\alpha\beta
 \frac{1\sum^{a}_{b}}{A^2}
\right\},\label{subeq:1}
\end{equation}
\end{subequations}
Giving a \verb+\label{#1}+ command directly after the \verb+\begin{subequations}+, 
allows you to reference all the equations in the \texttt{subequations} environment. 
For example, the equations in the preceding subequations environment were
Eqs.~(\ref{eq:whole}).

\subsubsection{Wide equations}

The equation:

\begin{widetext}
\begin{equation}
{\cal R}^{(\text{d})}=
 g_{\sigma_2}^e
 \left(
   \frac{[\Gamma^Z(3,21)]_{\sigma_1}}{Q_{12}^2-M_W^2}
  +\frac{[\Gamma^Z(13,2)]_{\sigma_1}}{Q_{13}^2-M_W^2}
 \right)
 + x_WQ_e
 \left(
   \frac{[\Gamma^\gamma(3,21)]_{\sigma_1}}{Q_{12}^2-M_W^2}
  +\frac{[\Gamma^\gamma(13,2)]_{\sigma_1}}{Q_{13}^2-M_W^2}
 \right)\;. 
 \label{eq:wideeq}
\end{equation}
\end{widetext}
This is typed to show how the output appears in wide format.


%\begin{acknowledgments}
%\end{acknowledgments}

\appendix

\section{Appendixes}


\section{A little more on appendixes}

\begin{equation}
E=mc^2.
\end{equation}


\bibliography{refs}

\end{document}